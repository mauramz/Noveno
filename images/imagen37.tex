%% Autor: Björn Ritterbecks 
%% Letzte Aenderung: 15.06.2016 
\thisfloatsetup{%
  capbesidewidth=\marginparwidth}
\begin{figure*}[htbp]
  \centering
\begin{pspicture}(-0.25,-0.25)(3,4.25)
%\psset{PointSymbol=none}
\pstTriangle[linewidth=1.2pt](0,0){A}(2.7,4){B}(2.7,0){C}
%\psset{PointName=none} 
\pstRightAngle[linewidth=1.2pt]{A}{C}{B}
\pstMarkAngle[LabelSep=1.15, MarkAngleRadius=.75]{A}{B}{C}{\(\alpha\)}
\pcline[linestyle=none](C)(A)
\Aput{\(27\) cm}
\pcline[linestyle=none](B)(C)
\Aput{\(40\) cm}
%\pcline[linestyle=none](S)(W)
%\Aput{z}
\end{pspicture}
  \caption[Triángulo ABC, ejercicio resuelto no. 5]{Triángulo rectángulo ABC, ejercicio resuelto no. 5.}
  \label{fig:imagen37}
  \vspace{-0pt}
\end{figure*}