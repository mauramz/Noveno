\thisfloatsetup{
  capbesidewidth=\marginparwidth,}
\begin{table}[htbp]
\centering
\sffamily
\small
%\sansmath
\arrayrulecolor{white}
\vspace{0.2cm}
  \rowcolors{2}{halfgray!15}{halfgray!5}
 \setlength{\extrarowheight}{.4em}
			\begin{tabularx}{0.99\textwidth}{l*{1}{>{\RaggedRight\arraybackslash}X}}		
\rowcolor{mycolor}\multicolumn{1}{l}{{\color{white}\textbf{Conocimientos}}}&  \multicolumn{1}{l}{{\color{white}\textbf{Habilidades}}}\\
\textbf{Números reales} & Identificar números irracionales en diversos contextos.\\
Números irracionales & Identificar números con expansión decimal infinita no periódica.\\
Concepto de número real & Realizar aproximaciones decimales de números irracionales. \\
Representaciones & Reconocer números irracionales en notación decimal, en notación radical y otras notaciones particulares.\\
Comparaciones & Comparar y ordenar números irracionales representados en notación decimal y radical.\\
Relaciones de orden & Identificar números reales (racionales e irracionales) y no reales en cualquiera de sus representaciones y en diversos contextos.\\
Recta numérica & Representar números reales en la recta numérica, con aproximaciones apropiadas.\\
		\end{tabularx}
		\caption[Tema Números Reales]{Conocimientos y Habilidades del tema Números Reales, tomadas del Programa de Estudio de Matemáticas del MEP} 
		\label{tab:cyhNumerosReales}
\vspace{0.2cm}		
\end{table}