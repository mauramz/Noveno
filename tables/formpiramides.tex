\thisfloatsetup{
  capbesidewidth=\marginparwidth,}
\begin{table}[htbp]
\centering
{\small
%\sansmath
\arrayrulecolor{white}
\renewcommand{\arraystretch}{1.5}
\vspace{0.2cm}
\rowcolors{2}{halfgray!15}{halfgray!5}
\setlength{\extrarowheight}{.4em}
\begin{tabular}{cccc}		
\rowcolor{mycolor}\color{white}{\textbf{Resultado}} & \color{white}{\textbf{Área Lateral}} & \color{white}{\textbf{Área Base}} & \color{white}{\textbf{Área Total}}\\
Base Triangular & \(3\cdot \dfrac{\ell \cdot A_p}{2}\) & \(\dfrac{\ell^2 \cdot \sqrt{3}}{2}\) &  \(3\cdot \dfrac{\ell \cdot A_p}{2}+\dfrac{\ell^2 \cdot \sqrt{3}}{2}\)\\
Base Cuadrada & \(2\cdot \ell \cdot A_p\) & \(\ell^2\) & \(2\cdot \ell \cdot A_p+\ell^2\) \\
Base Rectangular & \(a\cdot A_a + b\cdot A_b\) & \(a\cdot b\) & \(a\cdot A_a + b\cdot A_b + a\cdot b\) \\
\end{tabular}
}
		\caption[Fórmulas para áreas de Pirámides]{Fórmulas para las distintas áreas de la Pirámides.} 
		\label{tab:formespacio}
\vspace{0.2cm}
\end{table}