\thisfloatsetup{
  capbesidewidth=\marginparwidth,}
\begin{table}[htbp]
\centering
{\small
%\sansmath
\arrayrulecolor{white}
\renewcommand{\arraystretch}{1.5}
\vspace{0.2cm}
\rowcolors{2}{halfgray!15}{halfgray!5}
\setlength{\extrarowheight}{.4em}
\begin{tabular}{cccc}		
\rowcolor{mycolor}\color{white}{\textbf{Resultado}} & \color{white}{\textbf{Área Lateral}} & \color{white}{\textbf{Área Base}} & \color{white}{\textbf{Área Total}}\\
Triangular & \(3\cdot \ell \cdot h\) & \(\dfrac{\ell^2 \cdot \sqrt{3}}{2}\) &  \(3\cdot \ell \cdot h+\ell^2\cdot \sqrt{3}\)\\
Cuadrada & \(4\cdot \ell \cdot h\) & \(\ell^2\) & \(4\cdot \ell \cdot h+2\cdot \ell^2\) \\
Rectangular & \(2\left(a\cdot h + b\cdot h\right)\) & \(a\cdot b\) & \(2\left(a\cdot h + b\cdot h + a\cdot b\right)\) \\
\end{tabular}
}
		\caption[Fórmulas para áreas de Prismas]{Fórmulas para las distintas áreas de los Prismas.} 
		\label{tab:formprismas}
\vspace{0.2cm}
\end{table}