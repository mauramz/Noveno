\thisfloatsetup{
  capbesidewidth=\marginparwidth,}
\begin{table}[htbp]
\centering
\sffamily
\small
%\sansmath
\arrayrulecolor{white}
\vspace{0.2cm}
  \rowcolors{2}{halfgray!15}{halfgray!5}
 \setlength{\extrarowheight}{.4em}
		\begin{tabularx}{0.99\textwidth}{l*{1}{>{\RaggedRight\arraybackslash}X}}		
		\rowcolor{mycolor}\multicolumn{1}{l}{{\color{white}\textbf{Punto Original}}}&  \multicolumn{1}{l}{{\color{white}\textbf{Traslación}}}\\
	 	\(P(x,y)\) & \(P^\prime(x+a,y+b)\)\\
        \(a\) es el movimiento horizontal & \(b\) es el movimiento vertical\\
        & \\
        el número \(a\) y \(b\) es positivo & mueve a la derecha o arriba\\
        & \\
        el número \(a\) y \(b\) es negativo & mueve a la izquierda o abajo\\
		\end{tabularx}
		\caption[Fórmula de Traslación]{Fórmula de Traslación} 
		\label{tab:formtraslaciones}
\vspace{0.2cm}		
\end{table}