\thisfloatsetup{
  capbesidewidth=\marginparwidth,}
\begin{table}[htbp]
\centering
\sffamily
\small
%\sansmath
\arrayrulecolor{white}
\vspace{0.2cm}
  \rowcolors{2}{halfgray!15}{halfgray!5}
 \setlength{\extrarowheight}{.4em}
			\begin{tabularx}{0.99\textwidth}{l*{1}{>{\RaggedRight\arraybackslash}X}}		
\rowcolor{mycolor}\multicolumn{1}{l}{{\color{white}\textbf{Conocimientos}}}&  \multicolumn{1}{l}{{\color{white}\textbf{Habilidades}}}\\
\textbf{Triángulos} & Aplicar el teorema de Pitágoras en la resolución de problemas en diferentes contextos.\\
Teorema de Pitágoras & Encontrar la distancia entre dos puntos en el plano cartesiano, aplicando el teorema de Pitágoras.\\
	\end{tabularx}
		\caption[Tema Triángulos]{Conocimientos y Habilidades del tema Triángulos, tomadas del Programa de Estudio de Matemáticas del MEP} 
		\label{tab:cyhtriangulos}
\vspace{0.2cm}		
\end{table}