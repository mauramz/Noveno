\thisfloatsetup{
  capbesidewidth=\marginparwidth,}
\begin{table}[htbp]
\centering
\sffamily
\small
%\sansmath
\vspace{0.2cm}
\renewcommand{\arraystretch}{1.75}
  \rowcolors{2}{halfgray!15}{halfgray!5}
 \setlength{\extrarowheight}{.4em}
			\begin{tabularx}{0.99\textwidth}{l*{3}{>{\RaggedRight\arraybackslash}X}}		
\rowcolor{mycolor}\multicolumn{1}{l}{{\color{white}\textbf{Razón Trigonométrica}}} &  \multicolumn{1}{l}{{\color{white}\textbf{Símbolo}}} & \multicolumn{1}{l}{{\color{white}\textbf{Forma}}}\\
Seno & \(\sen \alpha\) & \(\dfrac{\text{Cateto Opuesto}}{\text{Hipotenusa}}\)\\
Coseno & \(\cos \alpha\) & \(\dfrac{\text{Cateto Adyacente}}{\text{Hipotenusa}}\)\\
Tangente & \(\tan \alpha\) & \(\dfrac{\text{Cateto Opuesto}}{\text{Cateto Adyacente}}\)\\
Cotangente & \(\cot \alpha\) & \(\dfrac{\text{Cateto Adyacente}}{\text{Cateto Opuesto}}\)\\
Secante & \(\sec \alpha\) & \(\dfrac{\text{Hipotenusa}}{\text{Cateto Adyacente}}\)\\
Cosecante & \(\csc \alpha\) & \(\dfrac{\text{Hipotenusa}}{\text{Cateto Opuesto}}\)\\
\end{tabularx}
		\caption[Razones Trigonométricas de un ángulo \(\alpha\)]{Razones Trigonométricas de un ángulo \(\alpha\).} 
		\label{tab:razones}
\vspace{0.2cm}		
\end{table}