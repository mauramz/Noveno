\thisfloatsetup{
  capbesidewidth=\marginparwidth,}
\begin{table}[htbp]
\centering
\sffamily
\small
%\sansmath
\vspace{0.2cm}
\renewcommand{\arraystretch}{1.5}
  \rowcolors{2}{halfgray!15}{halfgray!5}
 \setlength{\extrarowheight}{.4em}
			\begin{tabularx}{0.99\textwidth}{l*{4}{>{\RaggedRight\arraybackslash}X}}
\(\tan \alpha = \dfrac{\sen \alpha}{\cos \alpha}\) & \(\cot \alpha = \dfrac{1}{\tan\alpha}\) & \(\sec \alpha = \dfrac{1}{\cos \alpha}\) & \(\sen \alpha = \dfrac{1}{\csc \alpha}\)\\
\(\cot \alpha = \dfrac{\cos \alpha}{\sen \alpha}\) & \(\tan \alpha = \dfrac{1}{\cot\alpha}\) & \(\csc \alpha = \dfrac{1}{\sen \alpha}\) & \(\cos \alpha = \dfrac{1}{\sec \alpha}\)\\
\end{tabularx}
		\caption[Identidades trigonométricas de un ángulo \(\alpha\)]{Identidades trigonométricas de un ángulo \(\alpha\).} 
		\label{tab:razones}
\vspace{0.2cm}		
\end{table}