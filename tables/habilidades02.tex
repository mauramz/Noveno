\thisfloatsetup{
  capbesidewidth=\marginparwidth,}
\begin{table}[htbp]
\centering
\sffamily
\small
%\sansmath
\arrayrulecolor{white}
\vspace{0.2cm}
  \rowcolors{2}{halfgray!15}{halfgray!5}
 \setlength{\extrarowheight}{.4em}
\begin{tabularx}{0.99\textwidth}{l*{1}{>{\RaggedRight\arraybackslash}X}}		
\rowcolor{mycolor}\multicolumn{1}{l}{{\color{white}\textbf{Conocimientos}}}&  \multicolumn{1}{l}{{\color{white}\textbf{Habilidades}}}\\
\textbf{Cálculos y estimaciones} & Estimar el valor de la raíz de un número entero.\\
Suma & Determinar números irracionales con representación radical entre dos números enteros consecutivos.\\
Resta & Utilizar la calculadora para resolver operaciones con radicales. \\
Multiplicación & Reconocer números irracionales en notación decimal, en notación radical y otras notaciones particulares.\\
División & Comparar y ordenar números irracionales representados en notación decimal y radical.\\
Potencias & Identificar números reales (racionales e irracionales) y no reales en cualquiera de sus representaciones y en diversos contextos.\\
Radicales & Representar números reales en la recta numérica, con aproximaciones apropiadas.\\
\end{tabularx}
		\caption[Tema Cálculos y Estimaciones]{Conocimientos y Habilidades del tema Cálculos y Estimaciones tomadas del Programa de Estudio de Matemáticas del MEP} 
		\label{tab:cyhcalculosestimaciones}
\vspace{0.2cm}		
\end{table}