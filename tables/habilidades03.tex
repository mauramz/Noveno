\thisfloatsetup{
  capbesidewidth=\marginparwidth,}
\begin{table}[htbp]
\centering
\sffamily
\small
%\sansmath
\arrayrulecolor{white}
\vspace{0.2cm}
  \rowcolors{2}{halfgray!15}{halfgray!5}
 \setlength{\extrarowheight}{.4em}
			\begin{tabularx}{0.99\textwidth}{l*{1}{>{\RaggedRight\arraybackslash}X}}		
\rowcolor{mycolor}\multicolumn{1}{l}{{\color{white}\textbf{Conocimientos}}}&  \multicolumn{1}{l}{{\color{white}\textbf{Habilidades}}}\\
\begin{minipage}[c]{0.25\textwidth}\vspace{0.1in}\textbf{Cantidades muy grandes y muy pequeñas} \end{minipage}& Utilizar los prefijos del Sistema Internacional de Medidas para representar cantidades muy grandes y muy pequeñas.\\
 & Utilizar la calculadora o software de cálculo simbólico como recurso en la resolución de problemas que involucren las unidades.\\
		\end{tabularx}
		\caption[Tema Cant. muy grandes y muy pequeñas]{Conocimientos y Habilidades del tema Cantidades muy grandes y muy pequeñas, tomadas del Programa de Estudio de Matemáticas del MEP} 
		\label{tab:cyhCantidades}
\vspace{0.2cm}		
\end{table}