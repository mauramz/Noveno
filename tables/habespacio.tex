\thisfloatsetup{
  capbesidewidth=\marginparwidth,}
\begin{table}[htbp]
\centering
\sffamily
\small
%\sansmath
\arrayrulecolor{white}
\vspace{0.2cm}
  \rowcolors{2}{halfgray!15}{halfgray!5}
 \setlength{\extrarowheight}{.4em}
			\begin{tabularx}{0.99\textwidth}{l*{1}{>{\RaggedRight\arraybackslash}X}}		
\rowcolor{mycolor}\multicolumn{1}{l}{{\color{white}\textbf{Conocimientos}}}&  \multicolumn{1}{l}{{\color{white}\textbf{Habilidades}}}\\
\begin{minipage}[c]{0.4\textwidth}
\textbf{Geometría del Espacio}
\end{minipage} & \\
\begin{minipage}[c]{0.4\textwidth}
Pirámide recta\\
Apotema, área lateral y total
\end{minipage} & Identificar y calcular la apotema de pirámides rectas cuya base sea un cuadrado o un triángulo equilátero. Calcular el área lateral y el área total de una pirámide recta de base cuadrada, rectangular o triangular.\\
\begin{minipage}[c]{0.4\textwidth} 
\vspace{0.05in}
Prisma recto\\
Apotema, área lateral y total
\vspace{0.05in}
\end{minipage} & Calcular el área lateral y el área total de un prisma recto de base cuadrada, rectangular o triangular.\\
\end{tabularx}
		\caption[Tema Geometría del Espacio]{Conocimientos y Habilidades del tema Geometría del Espacio, tomadas del Programa de Estudio de Matemáticas del MEP} 
		\label{tab:cyhespacio}
\vspace{0.2cm}
\end{table}