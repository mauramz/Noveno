\thisfloatsetup{
  capbesidewidth=\marginparwidth,}
\begin{table}[htbp]
\centering
\sffamily
\small
%\sansmath
\arrayrulecolor{white}
\vspace{0.2cm}
  \rowcolors{2}{halfgray!15}{halfgray!5}
 \setlength{\extrarowheight}{.4em}
			\begin{tabularx}{0.99\textwidth}{l*{1}{>{\RaggedRight\arraybackslash}X}}		
\rowcolor{mycolor}\multicolumn{1}{l}{{\color{white}\textbf{Conocimientos}}}&  \multicolumn{1}{l}{{\color{white}\textbf{Habilidades}}}\\
\begin{minipage}[c]{0.4\textwidth}
\textbf{Trigonometría}\\ Radianes 
\end{minipage} & Convertir medidas angulares de grados a radianes y viceversa.\\
\begin{minipage}[c]{0.4\textwidth}
Seno, Coseno, Tangente	
\end{minipage} & Aplicar las razones trigonométricas básicas (seno, coseno, tangente) en diversos contextos. Aplicar las relaciones entre tangente, seno y coseno.\\
\begin{minipage}[c]{0.4\textwidth} 
\vspace{0.05in}
Razones trigonométricas de ángulos complementarios
\vspace{0.05in}
\end{minipage} & Aplicar tangente, seno y coseno de ángulos complementarios. Aplicar que la suma de los cuadrados del seno y coseno de un ángulo es 1.\\
\begin{minipage}[c]{0.4\textwidth}
Ángulos de elevación y depresión
\end{minipage}
 & Aplicar los conceptos de ángulos de elevación y depresión en diferentes contextos.\\
Ley de Senos & Aplicar la ley de senos en diversos contextos.\\
	\end{tabularx}
		\caption[Tema Trigonometría]{Conocimientos y Habilidades del tema Trigonometría, tomadas del Programa de Estudio de Matemáticas del MEP} 
		\label{tab:cyhtrigono}
\vspace{0.2cm}
\end{table}