%% Index (package imakeidx)
\IfDefined{makeindex}{%
 \IfPackageLoaded{imakeidx}{%
  \makeindex[%
    ,title=\indexname%
    ,program=makeindex% (makeindex,xindy,texindy)
    ,intoc=true,%
    ,columns=2%
    ,columnsep=35pt%
    ,columnseprule=false%
  ]%
 }%
}%
%% Glossary/Acronym list/list of symbols (glossaries package)
\IfDefined{makeglossaries}{\makenoidxglossaries}
% --| Index |---------------------------------------------------------

% prints 1st argument emphasized and indexes it
\newcommand{\emphidx}[1]{\emph{#1}\index{#1}}

% prints and indexes 1st argument
\newcommand{\idx}[1]{#1\index{#1}}

% --| Length |-------------------------------------------------------
% define margin width variable
\newlength{\marginwidth}
\setlength{\marginwidth}{\marginparwidth}
\addtolength{\marginwidth}{\marginparsep}

% define text width and height
\newlength{\doctextwidth}
\setlength{\doctextwidth}{\textwidth}
\newlength{\doctextheight}
\setlength{\doctextheight}{\textheight}
%% Glossary (depreciated glossary package - not supported by this template!)
\IfDefined{makenomenclature}{\makenomenclature}

%% Mini TOC (package minitoc - not supported by this template!)
%\IfPackageLoaded{minitoc}{\IfElseUndefined{chapter}{\dosecttoc}{\dominitoc}}
\contentsmargin{0cm}
\usepackage{mdframed}
\usepackage{amsthm}
%% Line numbers (package lineno)
%\IfDefined{linenumbers}{\linenumbers}
%% prints all new columntype definitions into the log file.
\IfDefined{showcols}{\showcols}


\newcommand{\intoo}[2]{\mathopen{]}#1\,;#2\mathclose{[}}
\newcommand{\ud}{\mathop{\mathrm{{}d}}\mathopen{}}
\newcommand{\intff}[2]{\mathopen{[}#1\,;#2\mathclose{]}}
\newtheorem{notation}{Notación}[chapter]
%%%%%%%%%%%%%%%%%%%%%%%%%%%%%%%%%%%%%%%%%%%%%%%%%%%%%%%%%%%%%%%%%%%%%%%%%%%
%%%%%%%%%%%%%%%%%%%% dedicated to boxed/framed environements %%%%%%%%%%%%%%
%%%%%%%%%%%%%%%%%%%%%%%%%%%%%%%%%%%%%%%%%%%%%%%%%%%%%%%%%%%%%%%%%%%%%%%%%%%
\newtheoremstyle{mycolornumbox}% % Theorem style name
{0pt}% Space above
{0pt}% Space below
{\normalfont}% % Body font
{}% Indent amount
{\bfseries\sffamily\color{mycolor}}% % Theorem head font
{\;}% Punctuation after theorem head
{0.25em}% Space after theorem head
{\sffamily\color{mycolor}\thmname{#1}\nobreakspace\thmnumber{#2}% Theorem text (e.g. Theorem 2.1)
\thmnote{\nobreakspace\the\thm@notefont\sffamily\bfseries\color{black}---\nobreakspace#3.}} % Optional theorem note
\renewcommand{\qedsymbol}{$\blacksquare$}% Optional qed square

\newtheoremstyle{blacknumex}% Theorem style name
{5pt}% Space above
{5pt}% Space below
{\normalfont}% Body font
{} % Indent amount
{\small\bfseries\sffamily}% Theorem head font
{\;}% Punctuation after theorem head
{0.25em}% Space after theorem head
{\small\sffamily
\nobreakspace\thmname{#1}\nobreakspace\thmnumber{#2}% Theorem text (e.g. Theorem 2.1)
\thmnote{\nobreakspace\the\thm@notefont\sffamily\bfseries\color{black}---\nobreakspace#3.}}% Optional theorem note

\newtheoremstyle{blacknumbox} % Theorem style name
{0pt}% Space above
{0pt}% Space below
{\normalfont}% Body font
{}% Indent amount
{\bfseries\sffamily}% Theorem head font
{\;}% Punctuation after theorem head
{0.25em}% Space after theorem head
{\sffamily\thmname{#1}\nobreakspace\thmnumber{#2}% Theorem text (e.g. Theorem 2.1)
\thmnote{\nobreakspace\the\thm@notefont\sffamily\bfseries\color{black}---\nobreakspace#3.}}% Optional theorem note
%%%%%%%%%%%%%%%%%%%%%%%%%%%%%%%%%%%%%%%%%%%%%%%%%%%%%%%%%%%%%%%%%%%%%%%%%%%
%%%%%%%%%%%%% dedicated to non-boxed/non-framed environements %%%%%%%%%%%%%
%%%%%%%%%%%%%%%%%%%%%%%%%%%%%%%%%%%%%%%%%%%%%%%%%%%%%%%%%%%%%%%%%%%%%%%%%%%
\newtheoremstyle{mycolornum}% % Theorem style name
{5pt}% Space above
{5pt}% Space below
{\normalfont}% % Body font
{}% Indent amount
{\bfseries\sffamily\color{mycolor}}% % Theorem head font
{\;}% Punctuation after theorem head
{0.25em}% Space after theorem head
{\sffamily\color{mycolor}\thmname{#1}\nobreakspace\thmnumber{#2}% Theorem text (e.g. Theorem 2.1)
\thmnote{\nobreakspace\the\thm@notefont\sffamily\bfseries\color{black}---\nobreakspace#3.}}% Optional theorem note
\renewcommand{\qedsymbol}{$\blacksquare$}% Optional qed square
\makeatother

% Defines the theorem text style for each type of theorem to one of the three styles above
\newcounter{dummy} 
\numberwithin{dummy}{chapter}
\theoremstyle{mycolornumbox}
\newtheorem{theoremeT}{Teorema}[chapter]
\newtheorem{problemeT}{Problema}[chapter]
\newtheorem{exampleT}{Ejemplo}[chapter]
\newtheorem{exerciseT}{Ejercicio}[chapter]
\newtheorem{definitionT}{Definición}[chapter]
\theoremstyle{blacknumex}
\newtheorem{vocabulary}{Vocabulario}[chapter]
\theoremstyle{blacknumbox}
\newtheorem{corollaryT}{Corolario}[chapter]
\theoremstyle{mycolornum}
\newtheorem{proposition}{Proposición}[chapter]

%----------------------------------------------------------------------------------------
%	DEFINITION OF COLORED BOXES
%----------------------------------------------------------------------------------------

% Required for creating the theorem, definition, exercise and corollary boxes

% Theorem box
\newmdenv[skipabove=7pt,
skipbelow=7pt,
backgroundcolor=black!5,
linecolor=mycolor,
innerleftmargin=5pt,
innerrightmargin=5pt,
innertopmargin=5pt,
leftmargin=0cm,
rightmargin=0cm,
innerbottommargin=5pt]{tBox}

% Exercise box	  
\newmdenv[skipabove=7pt,
skipbelow=7pt,
rightline=false,
leftline=true,
topline=false,
bottomline=false,
backgroundcolor=mycolor!10,
linecolor=mycolor,
innerleftmargin=5pt,
innerrightmargin=5pt,
innertopmargin=5pt,
innerbottommargin=5pt,
leftmargin=0cm,
rightmargin=0cm,
linewidth=4pt]{eBox}	

% Definition box
\newmdenv[skipabove=7pt,
skipbelow=7pt,
rightline=false,
leftline=true,
topline=false,
bottomline=false,
linecolor=mycolor,
innerleftmargin=5pt,
innerrightmargin=5pt,
innertopmargin=0pt,
leftmargin=0cm,
rightmargin=0cm,
linewidth=4pt,
innerbottommargin=0pt]{dBox}	

% Corollary box
\newmdenv[skipabove=7pt,
skipbelow=7pt,
rightline=false,
leftline=true,
topline=false,
bottomline=false,
linecolor=gray,
backgroundcolor=black!5,
innerleftmargin=5pt,
innerrightmargin=5pt,
innertopmargin=5pt,
leftmargin=0cm,
rightmargin=0cm,
linewidth=4pt,
innerbottommargin=5pt]{cBox}

% Creates an environment for each type of theorem and assigns it a theorem text style from the "Theorem Styles" section above and a colored box from above
\newenvironment{theorem}{\begin{tBox}\begin{theoremeT}}{\end{theoremeT}\end{tBox}}
\newenvironment{exercise}{\begin{eBox}\begin{exerciseT}}{\end{exerciseT}\end{eBox}}				  
\newenvironment{definition}{\begin{tBox}\begin{definitionT}}{\end{definitionT}\end{tBox}}	
\newenvironment{example}{\begin{cBox}\begin{exampleT}}{\end{exampleT}\end{cBox}}
\newenvironment{corollary}{\begin{cBox}\begin{corollaryT}}{\end{corollaryT}\end{cBox}}	
\newenvironment{problem}{\begin{cBox}\begin{problemeT}}{\end{problemeT}\end{cBox}}	
%----------------------------------------------------------------------------------------
%	REMARK ENVIRONMENT
%----------------------------------------------------------------------------------------

\newenvironment{remark}{% Vertical white space above the remark and smaller font size
\begin{list}{}{
\leftmargin=5pt % Indentation on the left
\rightmargin=10pt}\item\ignorespaces % Indentation on the right
\makebox[-2.5pt]{\begin{tikzpicture}[overlay]
\node[draw=mycolor!60,line width=2pt,circle,fill=mycolor!20,font=\LARGE\sffamily\bfseries,inner sep=2pt,outer sep=0pt] at (-15pt,0pt){\textcolor{mycolor}{!}};\end{tikzpicture}}
\advance\baselineskip -1pt}{\end{list}\vskip5pt} % Tighter line spacing and white space after remark

\newenvironment{duda}{% Vertical white space above the remark and smaller font size
\begin{list}{}{
\leftmargin=5pt % Indentation on the left
\rightmargin=10pt}\item\ignorespaces % Indentation on the right
\makebox[-2.5pt]{\begin{tikzpicture}[overlay]
\node[draw=mycolor!60,line width=2pt,circle,fill=mycolor!20,font=\LARGE\sffamily\bfseries,inner sep=2pt,outer sep=0pt] at (-15pt,0pt){\textcolor{mycolor}{?}};\end{tikzpicture}}
\advance\baselineskip -1pt}{\end{list}\vskip5pt} % Tighter line spacing and white space after remark

\newenvironment{youtube}{% Vertical white space above the remark and smaller font size
\begin{list}{}{
\leftmargin=5pt % Indentation on the left
\rightmargin=10pt}\item\ignorespaces % Indentation on the right
\makebox[-2.5pt]{\begin{tikzpicture}[overlay]
\node[draw=mycolor!60,line width=2pt,circle,fill=mycolor!20,font=\Large\sffamily\bfseries,inner sep=2pt,outer sep=0pt] at (-15pt,0pt){\textcolor{mycolor}{Y}};\end{tikzpicture}}
\advance\baselineskip -1pt}{\end{list}\vskip5pt} % Tighter line spacing and white space after remark


%----------------------------------------------------------------------------------------
%  MARGINS SECTIONTITLES
%----------------------------------------------------------------------------------------
\makeatletter
% Section numbers into left margin: llap = rlap, remove \hskip

%----------------------------------------------------------------------------------------
%	CHAPTER HEADINGS
%----------------------------------------------------------------------------------------
% A switch to conditionally include a picture, implemented by  Christian Hupfer
\newif\ifusechapterimage
\usechapterimagetrue
\newcommand{\thechapterimage}{}%
\newcommand{\chapterimage}[1]{\ifusechapterimage\renewcommand{\thechapterimage}{#1}\fi}%
\def\@makechapterhead#1{%
{\parindent \z@ \raggedright \normalfont
\ifnum \c@secnumdepth >\m@ne
\ifappendix % Falls das Kapitel auf einer rechten Seite beginnt
\startcontents
\begin{tikzpicture}[remember picture,overlay]
\node at (current page.north west)
{\begin{tikzpicture}[remember picture,overlay]
%\node[anchor=north west,inner sep=0pt] at (0,0) {\ifusechapterimage\includegraphics[width=\paperwidth]{\thechapterimage}\fi};
%\draw[anchor=west] (\Gm@lmargin-30mm,-4.5cm) node [line width=2pt,rounded corners=45pt,draw=mycolor,fill=white,fill opacity=0.5,inner sep=45pt]{\strut\makebox[24cm]{}};
%\draw[anchor=west][mycolor!50, thick](\Gm@lmargin+10mm,-5.9) -- (\Gm@lmargin+12.41cm,-5.9);
%\draw[anchor=west][mycolor!50, ultra thick](\Gm@lmargin+10mm,-4.3) -- (\Gm@lmargin+17.513cm,-5.7);
\draw[anchor=east] (13.2cm,-5cm) node
{\LARGE\color{black}\hfill\textsc{\textsc{\protect\parbox{1\textwidth}{\protect\raggedleft #1}}\strut}};
\draw[anchor=east] (17.8cm,-5.0cm) node {\fontseries{n}\fontsize{80}{130}\selectfont{\textcolor{sectioncolor} {{\fontencoding{U}\fontfamily{eur}\fontseries{b}\selectfont\@Alph\c@chapter}}\strut}};
\end{tikzpicture}};
\end{tikzpicture}
\else % Falls das Kapitel auf einer linken Seite beginnt
\begin{tikzpicture}[remember picture,overlay]
\node at (current page.north west)
{\begin{tikzpicture}[remember picture,overlay]
%-.\node[anchor=north west,inner sep=0pt] at (0,0) {\ifusechapterimage\includegraphics[width=\paperwidth]{\thechapterimage}\fi};
%\draw[anchor=west] (\Gm@lmargin-30mm,-4.5cm) node [line width=2pt,rounded corners=45pt,draw=mycolor,fill=white,fill opacity=0.5,inner sep=45pt]{\strut\makebox[24cm]{}};
%\draw[anchor=west][black!70, thick](\Gm@lmargin+10mm,-5.9) -- (\Gm@lmargin+12.41cm,-5.9);
%\draw[anchor=west][mycolor!50, ultra thick](\Gm@lmargin+10mm,-4.3) -- (\Gm@lmargin+17.513cm,-5.7);
\draw[anchor=east] (13.2cm,-5cm) node {\LARGE\color{black}\hfill\textsc{\textsc{\protect\parbox{.9\textwidth}{\protect\raggedleft #1}}\strut}};
\draw[anchor=east] (17.15cm,-5.0cm) node {\fontseries{n}\fontsize{80}{130}\selectfont{\textcolor{mycolor} {{\fontencoding{U}\fontfamily{eur}\fontseries{b}\selectfont\@arabic\c@chapter}}\strut}};
\end{tikzpicture}};
\end{tikzpicture}
\fi\fi\par\vspace*{90\p@}}}

% Beschneidung des Tufte-Plots
\newcommand\clipright[1][white]{
  \fill[#1](current axis.south east)rectangle(current axis.north-|current axis.outer east);
  \pgfresetboundingbox
  \useasboundingbox(current axis.outer south west)rectangle([xshift=.5ex]current axis.outer north-|current axis.east);
}

% Caption der Randabbildungen
\makeatletter
\DeclareCaptionStyle{marginfigure}{labelsep = colon, labelfont  = {mycolor, footnotesize, , it, sc, bf}, justification = RaggedRight, position=bottom, textfont = {footnotesize, it, color=halfgray}}
\DeclareCaptionStyle{margintable}{labelsep = colon, labelfont  = {mycolor, footnotesize, , it, sc, bf}, justification = RaggedRight, position=bottom, textfont = {footnotesize, it, color=halfgray}}

% Umdefinition der Listings-Bezeichnungen
\makeatletter
\renewcommand{\@marginparreset}{%
  \reset@font\scriptsize
  \RaggedRight
  \@setminipage
}
\lst@UserCommand\lstlistingname{R-code}
\lst@UserCommand\lstlistlistingname{Quellcode}
\makeatother

\usepackage{textcase} % provides \MakeTextUppercase and \MakeTextLowercase
% Define Lower Case \scshape
\newcommand\textlcsc[1]{\textsc{\MakeLowercase{#1}}} 


\newcounter{mynote}% a new counter for use in margin notes

\newcommand{\mynote}[2][0]{% a simple margin note
   \refstepcounter{footnote}% step counter
   \mbox{\textsuperscript{\tiny{\thefootnote}}}% the number (superscript) in text
   \marginnote{\mbox{\textsuperscript{\raisebox{-0.4mm}{\color{halfgray}\itshape\RaggedRight\thefootnote}}}{\color{halfgray}\textit{\RaggedRight#2}}}[#1\baselineskip]% the note
}

\deffootnote[0.4em]{0.0em}{0em}{% modified example from page 83
  \makebox[0.4em][l]{\textsuperscript{\raisebox{-0.4mm}{\itshape\thefootnotemark}}}}
  
 \makeatletter
 % Basically the same as for `\l@section` etc, just `\@nodottedtocline` instead of `\@dottedtcline`:
 \newcommand*\l@chapterinfo{\addvspace{1pt}\@nodottedtocline{1}{1.0cm}{0.0em}}
 \newcommand*\l@sectioninfo{\@nodottedtocline{1}{1.5em}{2.3em}}
 \newcommand*\l@subsectioninfo{\@nodottedtocline{2}{3.8em}{3.2em}}
 \newcommand*\l@subsubsectioninfo{\@nodottedtocline{3}{7.0em}{4.1em}}
 \newcommand*\l@paragraphinfo{\@nodottedtocline{4}{10em}{5em}}
 \newcommand*\l@subparagraphinfo{\@nodottedtocline{5}{12em}{6em}}
 
 
 % Copied from the book class macro `\@dottedtocline`. Removed the dots and page number
 \def\@nodottedtocline#1#2#3#4#5{%
   \ifnum #1>\c@tocdepth \else
     \vskip \z@ \@plus.2\p@
     {\leftskip #2\relax \parfillskip -\rightskip
      \parindent #2\relax\@afterindenttrue
      \interlinepenalty\@M
      \leavevmode
      \@tempdima #3\relax
      \advance\leftskip \@tempdima \null\nobreak\hskip -\leftskip
      {#4}\nobreak
      \leaders\hbox{$\m@th
         \mkern \@dotsep mu\hbox{\,}\mkern \@dotsep
         mu$}\hfill
      \nobreak
      \hb@xt@\@pnumwidth{\hfil\normalfont \normalcolor }%
      \par}%
   \fi}
  
 \makeatother
 
 \def\chapterinfo#1{%
     \addcontentsline{toc}{chapterinfo}{%
     \noexpand{}\color{halfgray}\itshape#1}%
 }
 \def\sectioninfo#1{%
     \addcontentsline{toc}{sectioninfo}{%
     \noexpand\numberline{}#1}%
 }
 \def\subsectioninfo#1{%
     \addcontentsline{toc}{subsectioninfo}{%
     \noexpand\numberline{}#1}%
 }
 %\renewcommand{\scdefault}{ssc} % MinionPro's spaced small caps as standard
 %\setstretch{1.025}
 %\setlength{\RaggedRightParindent}{\parindent}
 % Kleinbuchstaben
 \makeatletter
 \newcommand*{\Section}{}
 \let\Chapter\chapter
 \renewcommand*{\chapter}{%
   \@ifstar {\star@chapter}{\@dblarg\nonstar@chapter}%
 }
 \newcommand*{\star@chapter}[1]{%
   \Chapter*{\MakeLowercase{#1}}%
 }
 \newcommand*{\nonstar@chapter}[2][]{%
   \Chapter[{#1}]{\MakeLowercase{#2}}%
 }
 
 \let\Section\section
 \renewcommand*{\section}{%
   \@ifstar {\star@section}{\@dblarg\nonstar@section}%
 }
 \newcommand*{\star@section}[1]{%
   \Section*{\MakeLowercase{#1}}%
 }
 \newcommand*{\nonstar@section}[2][]{%
   \Section[{#1}]{\MakeLowercase{#2}}%
 }
 \let\Subsection\subsection
 \renewcommand*{\subsection}{%
   \@ifstar {\star@subsection}{\@dblarg\nonstar@subsection}%
 }
 \newcommand*{\star@subsection}[1]{%
   \Subsection*{\MakeLowercase{#1}}%
 }
 \newcommand*{\nonstar@subsection}[2][]{%
   \Subsection[{#1}]{\MakeLowercase{#2}}%
 }
% 
% \let\Paragraph\paragraph
% \renewcommand*{\paragraph}{%
%   \@ifstar {\star@paragraph}{\@dblarg\nonstar@paragraph}%
% }
% \newcommand*{\star@paragraph}[1]{%
%   \Paragraph*{\MakeLowercase{#1}}%
% }
% \newcommand*{\nonstar@paragraph}[2][]{%
%   \Paragraph[{#1}]{\MakeLowercase{#2}}%
% }
 \renewcommand\addchaptertocentry[2]{%
   \addtocentrydefault{chapter}{\protect\lowercase{#1}}{%
     \protect\texorpdfstring{\MakeLowercase{#2}}{#2}}}
     
     
     \captionsetup[capbesidefigure]{labelsep = colon, labelfont  = {mycolor, footnotesize, , it, sc, bf}, justification = RaggedRight, position=bottom, textfont = {footnotesize, it, color=halfgray}}
 %\renewcommand\addsectiontocentry[2]{%
 %  \addtocentrydefault{section}{\protect\lowercase{#1}}{%
 %    \protect\texorpdfstring{\MakeLowercase{#2}}{#2}}}    
 
 %\renewcommand\addsubsectiontocentry[2]{%
 %  \addtocentrydefault{subsection}{\protect\lowercase{#1}}{%
 %    \protect\texorpdfstring{\MakeLowercase{#2}}{#2}}}     
 \setcounter{tocdepth}{1}    
 % \bookmarklevel[3]{\chapterinfo} 
 \makeatother 
\renewcommand*{\raggedleftmarginnote}{\RaggedLeft}
\renewcommand*{\raggedrightmarginnote}{\RaggedRight}
\newcommand\Marginnote[1]{\marginnote{\hspace{0pt}#1}}

% --| other new definitions |-----------------------------------------



% --| Math |-------------------------------------------------------


% -- new commands --
\newcommand{\abs}[1]{\lvert#1\rvert}
\newcommand{\Abs}[1]{\left\lvert#1\right\rvert}
\newcommand{\norm}[1]{\left\Vert#1\right\Vert}
\newcommand{\Trace}[1]{\ensuremath{\Tr\left\{\,#1\,\right\}}} % Trace /Spur
%

% -- differentials --
\newcommand{\pd}{\partial\mspace{1mu}} % partial diff
\newcommand{\td}{\,\mathrm{d}}	% total diff

% -- Abbreviations --
\renewcommand{\Re}{\text{Re}}			% Real value
\renewcommand{\Im}{\text{Im}}			% Real value
\newcommand{\complex}{\mathbb{C}} % Complex
\newcommand{\real}{\mathbb{R}}    % Real
\renewcommand{\i}{\mathrm{i}}   
%
\newcommand{\Ham}{\mathcal{H}}    
\newcommand{\Prob}{\mathscr{P}}   
\newcommand{\unity}{\mathds{1}}   
%
% -- New Operators --
\IfDefined{DeclareMathOperator}{
  \DeclareMathOperator{\rot}{rot}
  \DeclareMathOperator{\grad}{grad}
  \DeclareMathOperator{\rect}{rect}
  %\renewcommand{\div}{\text{div}\,}
  \DeclareMathOperator{\Tr}{Tr}
  \DeclareMathOperator{\const}{const}
  \DeclareMathOperator{\e}{e} 			% exponatial Function
  
}

% -- new symbols --
\newcommand{\laplace}{\Delta}
\newcommand{\dalembert}{\Box}
\newcommand*{\tran}{^{\mkern-1.5mu\mathsf{T}}}

\newenvironment{items}{%
  \begin{list}{\textbullet}{%
    \setlength{\topsep}{0.6cm}
    \setlength{\partopsep}{0pt}
    \setlength{\parsep}{\parskip}
    \setlength{\itemsep}{\parskip}
    \setlength{\leftmargin}{0pt}
    \setlength{\itemindent}{0pt}
  }
}{\end{list}}

\newenvironment{enum}{%
  \begin{list}{\arabic{enumi}}{%
    \setlength{\topsep}{0.6cm}
    \setlength{\partopsep}{0pt}
    \setlength{\parsep}{\parskip}
    \setlength{\itemsep}{\parskip}
    \setlength{\leftmargin}{0pt}
    \setlength{\itemindent}{0pt}
    \usecounter{enumi}
  }
}{\end{list}}
