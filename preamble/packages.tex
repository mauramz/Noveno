% !TeX encoding=utf8
% !TeX program = pdflatex
% !TeX spellcheck = en-US

%% -- package section selections -->
\DefineCodeSection[true]{PackagesBase}
\DefineCodeSection[true]{PackagesBugfixes}
\DefineCodeSection[true]{PackagesFonts}
\DefineCodeSection[true]{PackagesDiagrams}
\DefineCodeSection[true]{PackagesMath}
\DefineCodeSection[true]{PackagesScience}
\DefineCodeSection[true]{PackagesSymbols}
\DefineCodeSection[true]{PackagesTables}
\DefineCodeSection[true]{PackagesText}
\DefineCodeSection[true]{PackagesQuotes}
\DefineCodeSection[true]{PackagesCitation}
\DefineCodeSection[true]{PackagesFigures}
\DefineCodeSection[true]{PackagesCaptions}
\DefineCodeSection[true]{PackagesIndexes}
\DefineCodeSection[true]{PackagesMisc}
\DefineCodeSection[true]{PackagesVerbatim}
\DefineCodeSection[true]{PackagesFancy}
\DefineCodeSection[true]{PackagesLayout}
\DefineCodeSection[true]{PackagesHeadFoot}
\DefineCodeSection[true]{PackagesHeadings}
\DefineCodeSection[true]{PackagesTOC}
\DefineCodeSection[true]{PackagesPDF}
\DefineCodeSection[true]{PackagesAdditional}
\BeginCodeSection{PackagesBase}

% Description: Calculation with LaTeX 
% Doc: calc.pdf
\usepackage{calc}

% Description: Multi Language support for LaTeX
% Doc: babel.pdf
\usepackage{babel}
% Description: support automatic translations
% Doc: beameruserguide.pdf
\usepackage{translator}


% Description: Color support with color mixing modells
% Doc: xcolor.pdf
\usepackage[
  x11names, % Load a set of predefined colors 
  table,      % Load the colortbl package
  % fixpdftex,  % Load the pdfcolmk package (may be problematic)
  hyperref,   % Support  the  hyperref  package
  fixinclude,
  cymk,
  natural % Prevent dvips color reset before .eps file inclusion
]{xcolor}
% Global color definitions
  \definecolor{violett}{cmyk}{0.7, 1, 0.35, 0.15} % RWTH violett
  \definecolor{mycolor2}{RGB}{73, 10, 61} % RWTH blau
  \definecolor{mycolor3}{RGB}{138,155,15}
  \definecolor{mycolor4}{RGB}{189,21,80}
  \definecolor{mycolor5}{RGB}{233,127,2}
  \definecolor{mycolor6}{RGB}{248,202,0}
  \definecolor{mycolor}{cmyk}{1, 0.5, 0, 0} %RWTH petrol
  \definecolor{bordeaux}{cmyk}{0.25, 1, 0.70, 0.20} % RWTH bordeaux
  %\definecolor{mycolor2}{cmyk}{35, 0, 100, 0} % RWTH mai-grün
  \definecolor{halfgray}{gray}{0.3}
  \definecolor{cobalt}{rgb}{0.0, 0.28, 0.67}

% Description: Support for graphics in LaTeX
% Doc: grfguide.pdf
\usepackage[%
  %final,
  %draft % do not include images (faster)
]{graphicx}

\usepackage[pdf]{pstricks}
\usepackage{pst-pdf, pst-node,pst-fill,pst-circ,pst-func,pst-math,pstricks-add,pst-slpe,pst-grad,pstricks,pst-node,pst-eucl}
% Description: If an eps image is detected, epstopdf is automatically 
%              called to convert it to pdf format.
% Requires: graphicx loaded
% Doc: epstopdf.pdf
\IfPackageLoaded{graphicx}{%
  \usepackage{epstopdf}
}


% Description:  environments for setting ragged text 
%               which allow hyphenation.
% Provides: \Centering, \RaggedLeft, and \RaggedRight, ... 
% Doc: ragged2e.pdf
\usepackage{ragged2e}
\usepackage{amsthm}
\usepackage{amssymb}
\EndCodeSection{PackagesBase}
% ~~~~~~~~~~~~~~~~~~~~~~~~~~~~~~~~~~~~~~~~~~~~~~~~~~~~~~~~~~~~~~~~~~~~~~~~
% LaTeX bug fixing packages
% ~~~~~~~~~~~~~~~~~~~~~~~~~~~~~~~~~~~~~~~~~~~~~~~~~~~~~~~~~~~~~~~~~~~~~~~~
\BeginCodeSection{PackagesBugfixes}

% Description: Fix known LaTeX2e bugs
% Doc: fixltx2e.pdf
%% Removed: fixltx2e is not required with releases after 2015
%%          All fixes are now in the LaTeX kernel.

% Description: This package implements a workaround for the LaTeX bug that
%              marginpars sometimes appear on the wrong margin.
% \usepackage{mparhack}
% BUG: in some case this causes an error in the index together with package
%      pdfpages the reason is unkown. Therefore I recommend to use the
%      margins of marginnote
% incompatible: marginfix

% Description: marginnote allows a margin note, where \marginpar fails 
% Doc: marginnote.pdf
\usepackage{marginnote}
\usepackage{sidenotes}

% Description: Redefines implementations of 
%              packages float, hyperref and listings
% Doc: scrhack.pdf
\usepackage{scrhack}

%% Description: changes the \marginpar commands, such
%%              that long margin notes work.
%% Doc: marginfix.pdf (TODO: why not used)
%\usepackage{marginfix}

% Description: Used to define commands that don't eat spaces.
% Doc: xspace.pdf
\RequirePackage{xspace}

\EndCodeSection{PackagesBugfixes}
% ~~~~~~~~~~~~~~~~~~~~~~~~~~~~~~~~~~~~~~~~~~~~~~~~~~~~~~~~~~~~~~~~~~~~~~~~
% Fonts
% ~~~~~~~~~~~~~~~~~~~~~~~~~~~~~~~~~~~~~~~~~~~~~~~~~~~~~~~~~~~~~~~~~~~~~~~~

\BeginCodeSection{PackagesFonts}

%% Description: Set the font size relative to the current font size
%% Doc: relsize-doc.pdf
\usepackage{relsize}
\usepackage{setspace}

\EndCodeSection{PackagesFonts}


\BeginCodeSection{PackagesMath}


% Description: basic math package
% Doc: amsldoc.pdf
\usepackage[
   centertags, % (default) center tags vertically
   %tbtags,    % 'Top-or-bottom tags': For a split equation, place equation
               % numbers level with the last (resp. first) line, if numbers
               % are on the right (resp. left).
   sumlimits,  %(default) Place the subscripts and superscripts of summation
               % symbols above and below
   %nosumlimits, % Always place the subscripts and superscripts of
                 % summation-type symbols to the side, even in displayed
                 % equations.
   intlimits,  % Like sumlimits, but for integral symbols.
   %nointlimits, % (default) Opposite of intlimits.
   namelimits, % (default) Like sumlimits, but for certain 'operator names'
               % such as det, inf, lim, max, min, that traditionally have
               % subscripts placed underneath when they occur in a displayed
               % equation.
   %nonamelimits, % Opposite of namelimits.
   %leqno,     % Place equation numbers on the left.
   %reqno,     % Place equation numbers on the right.
   % fleqn,      % Position equations at a fixed indent from the left margin
               % rather than centered in the text column.
]{amsmath} %

\IfPackageLoaded{amsmath}{

% Description: The mathtools package is an extension package to amsmath. 
%              Furthermore it corrects various bugs
% Doc: mathtools.pdf
\usepackage[fixamsmath,disallowspaces]{mathtools}

% Description: Inhibits the usage of plain TeX and 
%              of standard LaTeX math environments
% Doc: onlyamsmath.pdf
%\usepackage[
%  all,
%  % warning
%  error
%]{onlyamsmath}
% Note that many other packages have problems with the change of the 
% catcode of the $-char. Therefore workarounds/fixes for tikz and tabu
% are provided (loaded in style.tex)

} % end: IfPackageLoaded{amsmath}

% Description: Macros for Dirac bra-ket notation and sets.
% Doc: braket.pdf
\usepackage{braket}

% Description: strike out arguments in math mode
% Doc: cancel.sty
\usepackage{cancel}

%% Description: Emphasize equations
%% Doc: empheq.pdf
\usepackage{empheq} 
%\usepackage{MnSymbol}
%\usepackage[onlysansmath=true, retainmissing=true]{mdsymbol}
% Description: scales math mode output in all environments correct
% Doc: Mathmode.pdf
\usepackage{exscale} 

% Description: fixes for the default Computer Modern math fonts
% Doc: fixmath.pdf
\usepackage{fixmath}

% Description: Enables the correct use of the comma as 
%              a decimal separator in math mode
% Doc: icomma.pdf
\usepackage{icomma}

% Description: LaTeX 3 Package for nice inline fractions
% Provides: \sfrac{1}{2}
% Replaces: nicefrac
% Doc: xfrac.pdf 
\usepackage{xfrac} 

\EndCodeSection{PackagesMath}
% ~~~~~~~~~~~~~~~~~~~~~~~~~~~~~~~~~~~~~~~~~~~~~~~~~~~~~~~~~~~~~~~~~~~~~~~~
% diagrams
% ~~~~~~~~~~~~~~~~~~~~~~~~~~~~~~~~~~~~~~~~~~~~~~~~~~~~~~~~~~~~~~~~~~~~~~~~
\BeginCodeSection{PackagesDiagrams}

% tikz and pgf
% consumes at least one \write (more if external is used)
\usepackage{pgf}
\usepackage{tikz}
\IfPackageLoaded{pgf}{%
% \usepgflibrary{arrows}
}

\IfPackageLoaded{tikz}{%
%%% Chapter numbers according to 
%%% package version 2.10
%
%%% 12. Package, Environments, Scopes, and Styles
\usetikzlibrary{scopes}         % Shorthand for Scope Environments
\usetikzlibrary{intersections}  % Intersections of Arbitrary Paths
%%% 13. Specifying Coordinate
\usetikzlibrary{calc}           % Coordinate Calculations
%%% 14. Syntax for Path Specifications
%%% 15. Actions on Path
%%% 16. Nodes and Edge
\usetikzlibrary{positioning}    % Advanced Placement Options
%%% 17. Matrices and Alignment
%%% 18. Making Trees Grow
%%% 19. Plots of Function
%%% 20. Transparency
%%% 21. Decorated Path
% \usetikzlibrary{decorations}
%%% 22. Transformation
%%% 23. Arrow Tip Library
\usetikzlibrary{arrows}
%%% 24. Automata Drawing Library
% \usetikzlibrary{automata}
%%% 25. Background Library
\usetikzlibrary{backgrounds}
%%% 26. Calc Library -> see 13.
%%% 27. Calendar Library
%\usetikzlibrary{calendar}
%%% 28. Chains
% \usetikzlibrary{chains}
%%% 29. Circuit Libraries
% \usetikzlibrary{circuits}
% \usetikzlibrary{circuits.logic.IEC}
% \usetikzlibrary{circuits.ee.IEC}
%\usetikzlibrary{circuits.logic.US}
%%% 30. Decoration Library -> see 21.
%%% 31. Entity-Relationship Diagram Drawing Library
% \usetikzlibrary{er}
%%% 32. Externalization Library
% \usetikzlibrary{external} % uses \write, may fail
% \tikzexternalize % activate externalize! 
%%% 33. Fading Library
% \usetikzlibrary{fadings}
%%% 34. Fitting Library
\usetikzlibrary{fit}
%%% 35. Fixed Point Arithmetic Library
\usetikzlibrary{fixedpointarithmetic}
%%% 36. Floating Point Unit Library
\usetikzlibrary{fpu}
%%% 37. Lindenmayer System Drawing Library
%\usetikzlibrary{lindenmayersystems}
%%% 38. Matrix Library
% \usetikzlibrary{matrix}
%%% 39. Mindmap Drawing Library
%\usetikzlibrary{mindmap}
%%% 40. Paper Folding Diagrams Library
%\usetikzlibrary{folding}
%%% 41. Pattern Library
\usetikzlibrary{patterns}
%%% 42. Petri-Net Drawing Library
%\usetikzlibrary{petri}
%%% 43. Plot Handler Library (loaded autom.)
\usetikzlibrary{plothandlers}
%%% 44. Plot Mark Library
\usetikzlibrary{plotmarks}
%%% 45. Profiler Library
%%% 46. Shadings Library
\usetikzlibrary{shadings}
%%% 47. Shadow Library
% \usetikzlibrary{shadows}
%%% 48. Shape Library
% \usetikzlibrary{shapes.geometric}
% \usetikzlibrary{shapes.symbols}
% \usetikzlibrary{shapes.multipart}
% \usetikzlibrary{shapes.callouts}
% \usetikzlibrary{shapes.misc}
%%% 49. Spy Library: Magnifying Parts of Pictures
% \usetikzlibrary{spy}
%%% 50. SVG-Path Library
% \usetikzlibrary{svg.path}
%%% 51. To Path Library (loaded autom.)
\usetikzlibrary{topaths}
%%% 52. Through Library
% \usetikzlibrary{through}
%%% 53 Tree Library
% \usetikzlibrary{trees}
%%% 54 Turtle Graphics Library
% \usetikzlibrary{turtle}
}
%% added upon request of user 
\usepackage{tkz-base}
\usepackage{tkz-euclide}
%\usetkzobj{all}
\usepackage{tkz-fct}
%%

% pgfplots
\usepackage{pgfplots}
\usepackage{pgfplotstable}
\usetikzlibrary{pgfplots.patchplots}
\usetikzlibrary{pgfplots.dateplot}
\usetikzlibrary{pgfplots.colormaps}
\usetikzlibrary{pgfplots.groupplots}
\usetikzlibrary{pgfplots.polar}
\usetikzlibrary{pgfplots.units}

% Package imakeidx tests for \directlua and finds it defined, because it uses 
% eTeX's \ifdefined, however pgfplots redefines it to \relax. That causes
% an error in imakeidx.
% This is a workaround to make it work again. 
% However, this must be fixed in pgfplots, since it is a bug in that package.
\ifx\directlua\relax
  \let\directlua\undefinedBecauseOfBugInPgfplots
\fi

% Thanks to Heiko Oberdiek and Christian Feuersänger for providing this
% fix. See http://tex.stackexchange.com/questions/75049/error-at-ifnum-luatexversion68
% for more information % fix bug in pgfplots with \directlua

\EndCodeSection{PackagesDiagrams}
% ~~~~~~~~~~~~~~~~~~~~~~~~~~~~~~~~~~~~~~~~~~~~~~~~~~~~~~~~~~~~~~~~~~~~~~~~
% science packages
% ~~~~~~~~~~~~~~~~~~~~~~~~~~~~~~~~~~~~~~~~~~~~~~~~~~~~~~~~~~~~~~~~~~~~~~~~
\BeginCodeSection{PackagesScience}
 
% Description: upright symbols from euler package
%              [Euler] or Adobe Symbols [Symbol]
% Provides:    \upmu
% Doc: upgreek.pdf
%\usepackage[Symbolsmallscale]{upgreek} 
% --> Use only if the original font does not provide
%     the necessary upright symbols

% Description: Commands/symbols for both math and text mode
% Provides:    \degree, \celsius, \perthousand, \ohm, \micro
% Incompatible: siunitx
% Requires: Command \upmu
% \IfDefined{upmu}{\usepackage[upmu]{gensymb}}

% Description:  package for setting units in a 
%               typographically correct way.
% Incompatible: siunitx
%\usepackage{units}

% Description: siunitx aims to provide a unified method to
%              typeset numbers and units correctly and easily.
% Incompatible: gensymb, units
\IfPackagesNotLoaded{gensymb, units}{
  \usepackage{siunitx}
}{}

\EndCodeSection{PackagesScience}
% ~~~~~~~~~~~~~~~~~~~~~~~~~~~~~~~~~~~~~~~~~~~~~~~~~~~~~~~~~~~~~~~~~~~~~~~~
% Symbols
% ~~~~~~~~~~~~~~~~~~~~~~~~~~~~~~~~~~~~~~~~~~~~~~~~~~~~~~~~~~~~~~~~~~~~~~~~
\BeginCodeSection{PackagesSymbols}
%%% General Doc: symbols-a4.pdf
%
%% Math symbols
\IfPackagesNotLoaded{mathdesign,MnSymbol,MdSymbol}{
  \usepackage{dsfont}   %% Double Stroke Fonts
 % \usepackage{amssymb}
}{}
% Futher Math symbols and script fonts
\IfPackagesNotLoaded{MnSymbol,MdSymbol}{
  \usepackage{esint} % generate missing integrals for lmodern
  %
  % provides further symbols of the Text Companion (TC) fonts
  % such as \tcmu, \tcperthousand, \tcdegree
  \usepackage{mathcomp} 
 % \usepackage[mathcal]{euscript} %% adds euler mathcal font
  \IfPackagesNotLoaded{mdbch}{
    \usepackage{mathrsfs} % script font (\mathscr)
  }{}
}{}

%\usepackage[integrals]{wasysym}

%% The European Currency Symbol
\usepackage[gen]{eurosym}


%% Common Symbols
\usepackage{pifont}   %% ZapfDingbats

\EndCodeSection{PackagesSymbols}
% ~~~~~~~~~~~~~~~~~~~~~~~~~~~~~~~~~~~~~~~~~~~~~~~~~~~~~~~~~~~~~~~~~~~~~~~~
% Tables (Tabular)
% ~~~~~~~~~~~~~~~~~~~~~~~~~~~~~~~~~~~~~~~~~~~~~~~~~~~~~~~~~~~~~~~~~~~~~~~~
\BeginCodeSection{PackagesTables}

% Description:  some additional commands to enhance
%               the quality of tables
% Provides:     \toprule, \midrule, \bottomrule, \cmidrule
% Doc: booktabs.pdf
\usepackage{booktabs}

% Description: extends the standard tabular environment with cells
%              spanning over multiple rows.
% Doc: multirow.pdf
\usepackage{multirow, bigstrut}

% Description: Table spanning over many pages (from longtable package) 
%              and with strechable columns (from tabularx package)
% Doc: ltxtable.pdf 
% -> load afer hyperref 
\ExecuteAfterPackage{hyperref}{\usepackage{ltxtable}}

% Description: defines a single environment tabu to make all kinds of tabulars
%              It is more flexible than tabular, tabular*, tabularx and array
%              and extends the possibilities.
% Doc: tabu.pdf
\usepackage{tabu}

% tablestyles
\IfFileExists{tablestyles.sty}{
  \IfDefined{rowcolors}{\usepackage{tablestyles}}%
}{}


\EndCodeSection{PackagesTables}

% ~~~~~~~~~~~~~~~~~~~~~~~~~~~~~~~~~~~~~~~~~~~~~~~~~~~~~~~~~~~~~~~~~~~~~~~~
% text related packages
% ~~~~~~~~~~~~~~~~~~~~~~~~~~~~~~~~~~~~~~~~~~~~~~~~~~~~~~~~~~~~~~~~~~~~~~~~

\BeginCodeSection{PackagesText}

%%% bug fixing ===========================================
% description: fixes bug in ellipsis (...) 
% Doc: ellipsis.pdf
% -> load after babel
\usepackage[xspace]{ellipsis} 

%%% Text-decoration ======================================
%
% Description: commands for underlining for emphasis
% Provides: \ulin, \uuline, \sout, \xout, ...
% Doc: ulem.pdf
\usepackage[normalem]{ulem} 

% Description: commands for for emphasis
% Provides: \so, \ul, \st, ...
% Doc: soulutf8.pdf (loads soul.sty)
\usepackage{soulutf8}

% Description: enable linebreaks for URLs
% Provides: \url{}
% Doc: url.pdf
\usepackage[hyphens]{url}
\usepackage{apptools}

% Initiale
\usepackage{lettrine}
\usepackage{GoudyIn}

\renewcommand{\LettrineFontHook}{\color{mycolor!75}\GoudyInfamily{}}
\LettrineTextFont{\itshape}
\setcounter{DefaultLines}{3}%

% Landscape
\usepackage{pdflscape}

% Blindtext
\usepackage{lipsum}
%%% footnotes============================================

% Description: The footmisc package provides several different 
%              customisations of the way foonotes are represented.
%              Fixes a LaTeX bug with option 'bottom'
%
% Doc: footmisc.pdf
% Load after: setspace 
% Load before: hyperref
\ExecuteAfterPackage{setspace}{% 
%
\usepackage[%
   bottom,      % Footnotes appear always on bottom. This is necessary
                % especially when floats are used
   stable,      % Make footnotes stable in section titles
   perpage,     % Reset on each page
   %para,       % Place footnotes side by side of in one paragraph.
   side,       % Place footnotes in the margin
   ragged,      % Use RaggedRight
   marginal,
   norule,     % suppress rule above footnotes
   %hang,
   multiple,    % rearrange multiple footnotes intelligent in the text.
   %symbol,     % use symbols instead of numbers
]{footmisc}}


%% Description: footnotes are normally reset at each page.
%%              With this package they can be reset only at 
%%              defined headings, such as chapters.
%% Doc: chngcntr.pdf
% \usepackage{chngcntr}
% \counterwithout{footnote}{chapter}

%% Description: provides the command \tablefootnote to be used in
%%              a table or sidewaystable environment, 
%%              where \footnote will not work.
%% Doc: tablefootnote.pdf
%% Bug: does not work as expected, bug not found so far 
%% tablefootnote must be loaded after rotating
%\ExecuteAfterPackage{rotating}{%
% % and after hyperref
% \IfPackageNotLoaded{hyperref}{%
%  \ExecuteAfterPackage{hyperref}{%
%   \usepackage{tablefootnote}%
%  }%
% }{}%
%}%

%%% References ============================================
%
% Description:  provides \vref, which is similar to \ref but 
%               adds an additional page reference, like 
%               'on the facing page' or 'on page 27'
% Doc: varioref.pdf
\usepackage{varioref} 

% Description:  enhances  the cross-referencing  features,
%               allowing the format of cross-references to be determined
%               automatically according to the "type" of cross-reference
% Doc: cleveref.pdf
% loading: must be loaded after hyperref and after varioref
\ExecuteAfterPackage{hyperref}{
% caption and cleveref incompatible in Versions before 2011/12/24
  \usepackage{cleveref}[2011/12/24]
}

% Description: Extension of the xr package for
%              cross references, with hyperref support
% Doc: xr.pdf
% load: before hyperref
\usepackage{xr-hyper} 

%%% Lists ================================================
%
% Description: Allows the custom lists of type item, enum 
%              and description. It thereby replaces the packages
%              paralist, enumerate, mdwlist. 
% Incompatible: enumerate.
% Doc: enumitem.pdf
\IfPackageNotLoaded{enumerate}{
  \usepackage{enumitem}
}
%
%%% Other Environments ================================================
%
% Description: The abstract package provides control over the typesetting of
%              the abstract environment.
% Doc: abstract.pdf
\IfDefined{endabstract}{%
  \usepackage{abstract}
}

\EndCodeSection{PackagesText}

% ~~~~~~~~~~~~~~~~~~~~~~~~~~~~~~~~~~~~~~~~~~~~~~~~~~~~~~~~~~~~~~~~~~~~~~~~
% Quotes
% ~~~~~~~~~~~~~~~~~~~~~~~~~~~~~~~~~~~~~~~~~~~~~~~~~~~~~~~~~~~~~~~~~~~~~~~~
\BeginCodeSection{PackagesQuotes}
%
% Description: Advanced features for clever quotations
% Doc: csquotes.pdf
\usepackage[%
   babel,            % the style of all quotation marks will be adapted
                     % to the document language as chosen by 'babel'
   french=quotes,    % Styles of quotes in each language
  % english=british,
   german=guillemets
]{csquotes}

\EndCodeSection{PackagesQuotes}

% ~~~~~~~~~~~~~~~~~~~~~~~~~~~~~~~~~~~~~~~~~~~~~~~~~~~~~~~~~~~~~~~~~~~~~~~~
% figures, placement, floats and captions
% ~~~~~~~~~~~~~~~~~~~~~~~~~~~~~~~~~~~~~~~~~~~~~~~~~~~~~~~~~~~~~~~~~~~~~~~~
\BeginCodeSection{PackagesFigures}

%% Description: A package like "pst-pdf" for processing PostScript graphics
%%              with psfrag labels within pdfLaTeX documents.
%% Doc: pstool.pdf
%% DOES Not work together with the template!
%\usepackage[crop=pdfcrop]{pstool}

%% Description: provides new floats and enables H float modifier option
%%             (in future incompatible with Koma Script)
%% Doc: float.pdf
%% ---> replaced by floatrow package!
%\usepackage{float} 
\usepackage{subfig}

% Description: enables typesetting a narrow float at the edge of the text,
%              and making the text wrap around it. 
% load after: float
% load before: caption
% Provides: wrapfigure and wrapfloat
% Doc: wrapfig-doc.pdf
\usepackage{wrapfig}   

% Description: place floats after the reference
% Doc: no documentation
\usepackage{flafter}

% Description: Defines a \FloatBarrier command, beyond which floats may not
%              pass; useful, for example, to ensure all floats for a section
%              appear before the next \section command.
% Doc: placeins-doc.pdf
\usepackage[
  section    % "\section" command will be redefined with "\FloatBarrier"
]{placeins}
%

%% Description: Floating figures as in wrapfloat
%%              (old LaTeX2e package from 1996)
%% Doc: floatflt.pdf
% \usepackage{floatflt}

\EndCodeSection{PackagesFigures}
% ~~~~~~~~~~~~~~~~~~~~~~~~~~~~~~~~~~~~~~~~~~~~~~~~~~~~~~~~~~~~~~~~~~~~~~~~
% caption packages
% ~~~~~~~~~~~~~~~~~~~~~~~~~~~~~~~~~~~~~~~~~~~~~~~~~~~~~~~~~~~~~~~~~~~~~~~~
\BeginCodeSection{PackagesCaptions}

% Description: extents the float mechanism of LaTeX and
%              provides macros for precise placement of 
%              figures, tables and captions.
%              works well together with the caption pack.
% load before: caption 
% Doc: floatrow.pdf 
\let\tmp\newinsert
\let\newinsert\newbox
\usepackage{floatrow, fr-fancy}
\let\newinsert\tmp

% Description: The caption package offers customization
%              of captions in floating environments such
%              figure and table and cooperates with many 
%              other packages.
% Doc: caption.pdf (Required v3.2 or newer)
\usepackage{caption}[2011/08/06]

%% subfig ist NOT recommended, use subcaption instead
%% Incompatible: 
%% - loads package capt-of. Loading of 'capt-of' afterwards will fail therefor
%% - subcaption
%% loads: caption
%% Doc: subfig.pdf
%\usepackage{subfig} 

% Description: subcaption supports typesetting of sub-captions
%             (by using the the sub-caption feature of the caption package).
% incompatible: subfig
% Doc: subcaption.pdf
\IfPackageNotLoaded{subfig}{
  % load after caption package
  \usepackage{subcaption}[2011/08/17]
}

% Description: provides a margincap environment for putting 
%              captions into the outer document margin with 
%              either a top or bottom alignment.
% Doc: mcaption.pdf
\usepackage[
  top, %  vertical caption alignment (top, bottom)
]{mcaption}

% Description: provides two new environments, sidewaystable and sidewaysfigure,
%              and further commands to rotate content.
% Doc: rotating.pdf
\usepackage[figuresright]{rotating}

\EndCodeSection{PackagesCaptions}
% ~~~~~~~~~~~~~~~~~~~~~~~~~~~~~~~~~~~~~~~~~~~~~~~~~~~~~~~~~~~~~~~~~~~~~~~~
% misc packages
% ~~~~~~~~~~~~~~~~~~~~~~~~~~~~~~~~~~~~~~~~~~~~~~~~~~~~~~~~~~~~~~~~~~~~~~~~
\BeginCodeSection{PackagesMisc}

% Description: adds line numbers to the main text
% Doc: ulineno
%\usepackage[
%  ,left     %  margin placment (left, right, switch, switch*)
%  ,pagewise %  Number the lines from 1 on each page (pagewise, running)
%  ,modulo   %  Print line numbers only if they are multiples of five.
%]{lineno}

\EndCodeSection{PackagesMisc}
% ~~~~~~~~~~~~~~~~~~~~~~~~~~~~~~~~~~~~~~~~~~~~~~~~~~~~~~~~~~~~~~~~~~~~~~~~
% Index and other lists
% ~~~~~~~~~~~~~~~~~~~~~~~~~~~~~~~~~~~~~~~~~~~~~~~~~~~~~~~~~~~~~~~~~~~~~~~~
\BeginCodeSection{PackagesIndexes}

%% Description: print text of \index{entry} to the margin
%% Doc: makeidx.pdf
%% --> load only in draft mode
%% load before: imakeidx
\IfDraft{
  \usepackage{showidx}
}


%% Description makeindex package with shell-escape makeindex call
%% Doc: imakeidx.pdf
% consumes \write
\usepackage{imakeidx}

%% Description: Package for glossaries, nomenclatures and acronym lists
%% replaces: nomencl, acronym
%% load after: hyperref!, inputenc, babel and ngerman.
% consumes \write (1 in general, 2 if entries are defined inside the document)
\ExecuteAfterPackage{hyperref}{%
\usepackage[
%%% General Options
  % nomain, % This suppresses the creation of the main glossary and associated
          % .glo file, if unrequired. Note that if you use this option,
          % you must create another glossary in which to put all your
          % entries (either via the acronym (or acronyms) package option
  % sanitizesort, % This is a boolean option that determines whether or not
                % to sanitize the sort value when writing to the external glossary
                % file.          
  %savewrites, % This is a boolean option to minimise the number of
              % write registers used by the glossaries package. 
              % (Default is savewrites=false.)
              % savewrites
			  % Note!: This option can significantly slow document compilation. 
			  % As an alternative, you can use the scrwfile package and not use this option.
			  % -> scrwfile disabled because of incompatibility with titletoc.
  translate=true, % If babel has been loaded and the translator package
                  % is installed, translator will be loaded and the translations
                  % will be provided by the translator package interface.
  hyperfirst=true, % options: (*true*, false)
                  % This is a boolean option that specifies whether each term
                  %  has a hyperlink on first use.
%
%%% Sectioning, Headings and TOC Options
  % toc,          % Add the glossaries to the table of contents.
  numberline,     % When used with toc, this will add \numberline{} in
                  % the final argument of \addcontentsline. This will align the
                  % table of contents entry with the numbered section titles.
  section=section, % Its value should be the name of a sectional unit (e.g. chapter). 
                  % This will make the glossaries appear in the named sectional unit, 
                  % otherwise each glossary will appear in a chapter, 
                  % if chapters exist, otherwise in a section.                  
  numberedsection = false,%
  	% The glossaries are placed in unnumbered sectional
  	% units by default, but this can be changed using numberedsection.
  	% options
  	% - false: no number, i.e. use starred form of sectioning command
  	% - nolabel: use a numbered section, but the section not labelled
  	% - autolabel: numbered with automatic labelling.
%
%%%  Glossary Appearance Options
  % entrycounter=false % (true, *false*)
                       % If set, each main (level 0) glossary entry will
                       % be numbered when using the standard glossary styles.
  % counterwithin=0 % if set will reset the glossaryentry counter every
                    % time the defined level is reset. 
  % nolong,  % prevents loading of glossary-long and thus the longtable package                 
  % nosuper, % prevents loading of glossary-super and thus the supertabular package
  % nolist,  % prevents loading of glossary-list
  % notree,  % prevents loading of glossary-tree
  nonumberlist, %  This option will suppress the 
                % associated number lists in the glossaries
  counter=page, % The value should be the name of the default counter 
                % to use in the number lists ).
%%% Sorting Options
  sort=standard,%
    % options
    % - standard : entries are sorted according to the value of the
    %              sort key used in \newglossaryentry (if present) 
    %              or the name key (if sort key is missing);
    % - def : entries are sorted in the order in which they were defined
    % - use : entries are sorted according to the order in which they
    %         are used in the document 
%%% Acronym Options    
  acronym,    % Creates a separate acronym list
  shortcuts,  % define shortcuts (\ac for acronym)
]{glossaries}
% further styles
\usepackage{glossary-longragged}
% Create a new list of symbols
\newglossary[slg]{symbolslist}{syi}{syg}{List of Symbols}
% Simplest and easiest sorting method, but it's
% inefficient and the sorting is done according to the English alphabet. To
% use this method, add \makenoidxglossaries to the preamble and put
% \printnoidxglossaries at the place where you want your glossary
%\makenoidxglossaries
}

\EndCodeSection{PackagesIndexes}\usepackage{eso-pic}
% ~~~~~~~~~~~~~~~~~~~~~~~~~~~~~~~~~~~~~~~~~~~~~~~~~~~~~~~~~~~~~~~~~~~~~~~~
% verbatim packages
% ~~~~~~~~~~~~~~~~~~~~~~~~~~~~~~~~~~~~~~~~~~~~~~~~~~~~~~~~~~~~~~~~~~~~~~~~
\BeginCodeSection{PackagesVerbatim}
%%% Doc: upquote.sty
\usepackage{upquote} % print correct quotes in verbatim-environments

% Description: Reimplementation of the original verbatim enironment
% Doc: verbatim.pdf
\usepackage{verbatim} %

% Description: This package provides many facilities for reading, writing and
%              changing the output style of verbatim code
% Doc: fancyvrb.pdf
% consumes \write
\usepackage{fancyvrb} 

% Description: The listings package is a source code printer for LaTeX.
%              You can typeset stand alone files as well as listings with an 
%              environment.
%              If the Syntax Highlighting of the preferred  programming
%              language is not already supported, you can make your own
%              definition.
% Doc: listings.pdf
% consumes \write
\usepackage{listings}



\EndCodeSection{PackagesVerbatim}
\usepackage[scaled=0.82]{beramono}
% ~~~~~~~~~~~~~~~~~~~~~~~~~~~~~~~~~~~~~~~~~~~~~~~~~~~~~~~~~~~~~~~~~~~~~~~~
% fancy packages
% ~~~~~~~~~~~~~~~~~~~~~~~~~~~~~~~~~~~~~~~~~~~~~~~~~~~~~~~~~~~~~~~~~~~~~~~~
\BeginCodeSection{PackagesFancy}

% Description: Dropping capitals
% Doc: lettrine.pdf
\usepackage{lettrine}

% Doc: boxedminipage.pdf
\usepackage{boxedminipage}

% Description: Create framed, shaded, or differently highlighted 
%              regions that can break across pages. 
% Doc: framed.pdf
% --> replaced by mdframed (take out ???)
\usepackage{framed}

% Description: defines new environments where the user may choose 
%              between several individual designs.
% Doc: mdframed-doc-en.pdf
\usepackage{mdframed}

\EndCodeSection{PackagesFancy}
% ~~~~~~~~~~~~~~~~~~~~~~~~~~~~~~~~~~~~~~~~~~~~~~~~~~~~~~~~~~~~~~~~~~~~~~~~
% layout packages
% ~~~~~~~~~~~~~~~~~~~~~~~~~~~~~~~~~~~~~~~~~~~~~~~~~~~~~~~~~~~~~~~~~~~~~~~~
\BeginCodeSection{PackagesLayout}

%%% indentation =========================================

% Description: Indent first paragraph after section header
% Doc: indentfirst.pdf
% \usepackage{indentfirst}

%%% columns =============================================

% Description: Environment for multicolumn text
% Doc: multicol.pdf
\usepackage{multicol}

% Subfigures gescheit referenzieren
\makeatletter
\renewcommand\p@subfigure{\thefigure\,}
\renewcommand\thesubfigure{(\alph{subfigure})}
\DeclareCaptionLabelFormat{mysublabelfmt}{(\alph{sub\@captype})}
\makeatother
%% line spacing =========================================
%
% Description: configure line spacing
% Provides: \onehalfspacing, \doublespacing
% Doc: setspace.sty
\usepackage{setspace}

%% page layout ==========================================

%% Test the page layout
%% Doc: layman.pdf
%\usepackage{layouts}
%\newcommand\showpage{%
%\setlayoutscale{0.27}\setlabelfont{\tiny}%
%\printheadingsfalse\printparametersfalse
%\currentpage\pagedesign}

% Layout with 'typearea' 
% -> loaded automatically if geometry not loaded
% Doc: scrguide.pdf

% Description: Margin adjustment and detection of odd/even pages.
% Doc: changepage.pdf
 \usepackage[strict]{changepage}

\EndCodeSection{PackagesLayout}
% ~~~~~~~~~~~~~~~~~~~~~~~~~~~~~~~~~~~~~~~~~~~~~~~~~~~~~~~~~~~~~~~~~~~~~~~~
% head and foot lines
% ~~~~~~~~~~~~~~~~~~~~~~~~~~~~~~~~~~~~~~~~~~~~~~~~~~~~~~~~~~~~~~~~~~~~~~~~
\BeginCodeSection{PackagesHeadFoot}

%%% Doc: scrguide.pdf
\usepackage[%
%%% Lines
   % headtopline,
   % plainheadtopline,
   % headsepline,
   % plainheadsepline,
   % footsepline,
   % plainfootsepline,
   % footbotline,
   % plainfootbotline,
   % ilines,
   % clines,
   % olines,
% column titles (content, style)
%   automark,
%   autooneside,% ignore optional argument in automark at oneside
%   komastyle,
   % standardstyle,
   % markuppercase,
   % markusedcase,
   nouppercase,
]{scrlayer-scrpage}

%\usepackage{fancyhdr} % Required for header and footer configuration

% Description: provides total number of pages (ie. page 7 of 19)
% Provides: \lastpageref{LastPage}
% load after: hyperref
% Doc: pageslts.pdf
%\ExecuteAfterPackage{hyperref}{\usepackage{pageslts}}

\EndCodeSection{PackagesHeadFoot}
% ~~~~~~~~~~~~~~~~~~~~~~~~~~~~~~~~~~~~~~~~~~~~~~~~~~~~~~~~~~~~~~~~~~~~~~~~
% layout of headings 
% ~~~~~~~~~~~~~~~~~~~~~~~~~~~~~~~~~~~~~~~~~~~~~~~~~~~~~~~~~~~~~~~~~~~~~~~~
\usetikzlibrary{arrows.meta}
\BeginCodeSection{PackagesHeadings}

% Description: The titlesec package is essentially a replacement - partial or
%              total-for the LaTeX macros related with sections - namely
%              titles, headers and contents.
%%% Doc: titlesec.pdf
%\ifcsdef{chapter} 
%	{\usepackage{titlesec}}
%	{\usepackage{titlesec} \csundef{chapter}}
%\EndCodeSection{PackagesHeadings}

% ~~~~~~~~~~~~~~~~~~~~~~~~~~~~~~~~~~~~~~~~~~~~~~~~~~~~~~~~~~~~~~~~~~~~~~~~
% settings and layout of TOC
% ~~~~~~~~~~~~~~~~~~~~~~~~~~~~~~~~~~~~~~~~~~~~~~~~~~~~~~~~~~~~~~~~~~~~~~~~

\BeginCodeSection{PackagesTOC}

% Description: The philosophy of this package is to use new commands which you
%              can format the toc entries with in a generic way.
% Doc: titlesec.pdf
% load before: hyperref
% consumes \write
% usage: % Define partial toc for part pages \PartialToc
\usepackage{titletoc}
% Description: apply different styles for the formating of the 
%              table of contents and lists of floats.
%%% Doc: tocstyle.pdf (Koma Script)
%% Alpha package, uses koma fonts (\setkomafont{}{}) only if KOMAlike is selected
%
%\usepackage[%
%%% toc width calculation 
%  tocindentauto,     % all widths at the TOCs are calculated by tocindentauto
%  tocindentmanual,  % opposite of auto
%%% indentation of toc
%  tocgraduated,      % standard
%  tocflat,          % no intendation, text aligned
%  tocfullflat,      % no intendation, no alignment
%%%  page breaking rules
%  tocbreaksstrict,   % sets a lot of penalties before and after TOC entries 
                     % to avoid page break between a TOC entry and it's parent. 
%  tocbreakscareless,% allow more page breaks.  
%%%  indentation of unnumbered TOC entries
% toctextentriesindented, % unnumbered TOC entrie are indented only as wide 
%                         % as the number of numbered TOC entries of the same 
%                         % level. 
%  toctextentriesleft,   % indented as if they have an empty number.
%]{tocstyle}

% Description: The appendix package provides some facilities for 
%              modifying the typesetting of appendix titles.
% Doc: appendix.pdf
%\usepackage[
% ,toc   % Put a header (e.g., 'Appendices') into the Table of Contents
% %,page  % Puts a title  (e.g.,  'Appendices') into the document at the 
%        % beginning of the appendices environment
% %,title % Adds a name (e.g., 'Appendix') before each appendix title in
%        % the body of the document.
% %,titletoc % Adds a name (e.g., 'Appendix') before each appendix listed 
%        % in the ToC
% %,header% Adds a name (e.g., 'Appendix') before each appendix in page headers.
%]{appendix}
%\renewcommand{\appendixtocname}{\appendixname}

\EndCodeSection{PackagesTOC}
% ~~~~~~~~~~~~~~~~~~~~~~~~~~~~~~~~~~~~~~~~~~~~~~~~~~~~~~~~~~~~~~~~~~~~~~~~
% pdf packages
% ~~~~~~~~~~~~~~~~~~~~~~~~~~~~~~~~~~~~~~~~~~~~~~~~~~~~~~~~~~~~~~~~~~~~~~~~

\BeginCodeSection{PackagesPDF}

% Description: Include pages from external PDF documents in LaTeX documents
% Doc: pdfpages.pdf
\usepackage{pdfpages} 

% Description: landscape orientation in PDF Format
% Doc: pdflscape.pdf
% load after: footmisc (correct ?)
%\usepackage{pdflscape}

% Description: The microtype package provides a LaTeX interface to the  
%              micro-typographic extensions of pdfTEX: most prominently,
%              character protrusion and font expansion, furthermore
%              the adjustment of interword spacing and additional kerning.
% Provides:    Much better textformating and better typography, 
%              but at the cost of a much larger PDF file.
% Doc: microtype.pdf

\usepackage[protrusion=true]{microtype}

% Description: add hyperlink support to LaTeX
% load: after almost every package!
% Doc: manual.pdf
\usepackage[
%%% Extension options
  ,backref=page       % Adds backlink text to the end of each item in the
                      % bibliography, as a list of section numbers.
                      % (section, slide, page, none)
  ,pagebackref=false  % Adds backlink text to the end of each item in the
                      % bibliography, as a list of page numbers.
  ,hyperindex=true    % Makes the page numbers of index entries into
                      % hyperlinks.
  ,hyperfootnotes=false % Makes the footnote marks into hyperlinks to the
                        % footnote text (must be false if footmisc is loaded).
%%% PDF-specific display options
  ,bookmarks=true
  ,bookmarksdepth=0
%%% PDF display and information options  
  ,pdfpagelabels=true % set PDF page labels
]{hyperref}

% Description: This package implements a new bookmark (outline) organization
%              for package  hyperref. In contrast to hyperref here only one 
%              LaTeX run is required.
% load: after hyperref
% Doc: bookmark.pdf
\IfNotDraft{%
  \usepackage{bookmark}
}

\usepackage[margincaption,outercaption,ragged,wide]{sidecap}
\sidecaptionvpos{figure}{t} 
\sidecaptionvpos{table}{t}

\EndCodeSection{PackagesPDF}
% ~~~~~~~~~~~~~~~~~~~~~~~~~~~~~~~~~~~~~~~~~~~~~~~~~~~~~~~~~~~~~~~~~~~~~~~~
% additional packages 
% ~~~~~~~~~~~~~~~~~~~~~~~~~~~~~~~~~~~~~~~~~~~~~~~~~~~~~~~~~~~~~~~~~~~~~~~~
% All packages added here MUST be loadeable after hyperref!
% ~~~~~~~~~~~~~~~~~~~~~~~~~~~~~~~~~~~~~~~~~~~~~~~~~~~~~~~~~~~~~~~~~~~~~~~~

\BeginCodeSection{PackagesAdditional}

% Description: enable hyphenation of typewriter text word (\texttt)
% Doc:  hyphenat.pdf
% Note: According to documentation the font warnings can be ignored
%\usepackage[htt]{hyphenat}
\usepackage{cmap} 

% T1 Schrift Encoding
%\usepackage[T1]{fontenc} 

% Description: Additional Symbols (Text Companion font extension)
% Doc: encguide.pdf

\usepackage{textcomp}   

% DO NOT LOAD ae Package as a font !

%\input{fonts/fonts-MyriadPro.tex}  %% --- MyriadPro
%\renewcommand{\rmdefault}{MinionPro}
\usepackage[%
  % disable,
]{todonotes}

\usepackage[NoDate]{currvita}

% \usepackage{nicefilelist}
\usepackage{ifthen}

\EndCodeSection{PackagesAdditional}
\usepackage{sansmath}

%\usepackage[onlysansmath=true]{mdsymbol}
%\usepackage[mathlf, mnsy, footnotefigures, minionint]{MinionPro}%% --- MinionPro
%\usepackage[scaled=.92]{avant}
\usepackage[toc]{tabfigures}

%\renewcommand{\sfdefault}{Myriad-LF}
%
%\usepackage[scaled=.92]{helvet}
%
%\usepackage{sansmathfonts}
% set bold to medium bold by default
%\renewcommand{\bfdefault}{sb}
% ~~~~~~~~~~~~~~~~~~~~~~~~~~~~~~~~~~~~~~~~~~~~~~~~~~~~~~~~~~~~~~~~~~~~~~~~
% last package
% ~~~~~~~~~~~~~~~~~~~~~~~~~~~~~~~~~~~~~~~~~~~~~~~~~~~~~~~~~~~~~~~~~~~~~~~~
% This package only indicates the last package loaded.
% It provides no functionality, it is just used by the command
% \ExecuteAfterPackage{lastpackage} to execute code before
% parameters of packages are set.
\usepackage{lastpackage}
